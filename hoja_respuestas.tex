\documentclass[letterpaper]{article}
\usepackage[top=1.1cm, bottom=1.0cm, left=1.3cm, right=1.2cm]{geometry}
\usepackage{tikz}
\usetikzlibrary{calc}
\usepackage{multicol}
\usepackage[T1]{fontenc}
\usepackage[utf8]{inputenc}
\usepackage{helvet}
\renewcommand{\familydefault}{\sfdefault}
\setlength{\parindent}{0pt}
\setlength{\parskip}{0pt}
\pagestyle{empty}

% ─── BURBUJA OMR ──────────────────────────────────────────────────────────────
% Círculo gris muy claro — el estudiante lo rellena completamente con lápiz
\newcommand{\B}{%
  \tikz[baseline=-2pt]{%
    \filldraw[fill=black!10, draw=black!50, line width=0.5pt]
      (0,0) circle (4.2pt);}%
  \hspace{2.5pt}%
}

% ─── TIMING MARK ──────────────────────────────────────────────────────────────
% Rectángulo negro sólido al inicio de cada fila.
% El algoritmo lo detecta para encontrar la Y exacta de cada pregunta.
\newcommand{\TM}{%
  \tikz[baseline=-3pt]{%
    \fill[black] (0,-4pt) rectangle (6pt,4pt);}%
  \hspace{3pt}%
}

% ─── FILA DE PREGUNTA ─────────────────────────────────────────────────────────
\newcommand{\Q}[1]{%
  \noindent\TM{\scriptsize\textbf{#1.}}\hspace{1.5pt}\B\B\B\B\par\vspace{1pt}%
}

% ─── ENCABEZADO A B C D ───────────────────────────────────────────────────────
\newcommand{\HDR}{%
  \noindent\hspace{12pt}%
  {\tiny\textbf{A}}\hspace{5.5pt}%
  {\tiny\textbf{B}}\hspace{5.5pt}%
  {\tiny\textbf{C}}\hspace{5.5pt}%
  {\tiny\textbf{D}}\par\vspace{2pt}%
}

% ─── BURBUJA PEQUEÑA (para grilla de ID) ──────────────────────────────────────
\newcommand{\Bs}{%
  \tikz[baseline=-1.5pt]{%
    \filldraw[fill=black!10, draw=black!50, line width=0.4pt]
      (0,0) circle (3.3pt);}%
}

% ─── COLUMNA DE DÍGITO (para grilla de identificación) ───────────────────────
% Cada columna muestra el número del dígito y 10 burbujas (0-9)
\newcommand{\DIG}[1]{%
  \begin{minipage}[t]{11pt}\centering
    \vspace{0pt}{\tiny\textbf{#1}}\par\vspace{1pt}
    \Bs\par\vspace{0.5pt}%  0
    \Bs\par\vspace{0.5pt}%  1
    \Bs\par\vspace{0.5pt}%  2
    \Bs\par\vspace{0.5pt}%  3
    \Bs\par\vspace{0.5pt}%  4
    \Bs\par\vspace{0.5pt}%  5
    \Bs\par\vspace{0.5pt}%  6
    \Bs\par\vspace{0.5pt}%  7
    \Bs\par\vspace{0.5pt}%  8
    \Bs\par%                9
  \end{minipage}%
  \hspace{2pt}%
}

% ─── MARCADOR FIDUCIAL (cuadrado negro sólido) ────────────────────────────────
% Se colocan en las 4 esquinas de la hoja para corregir perspectiva.
% Tamaño: 7×7 mm  — offset 8 mm desde el borde
\newcommand{\fiducial}[2]{% #1=xshift, #2=yshift (con unidades)
  \begin{tikzpicture}[remember picture, overlay]
    \fill[black] (current page.#1) ++#2 rectangle ++(7mm,-7mm);
  \end{tikzpicture}%
}

\begin{document}

% ═══════════════════════════════════════════════════════════════════════════════
%  MARCADORES FIDUCIALES — 4 esquinas
%  Colocados con tikzpicture separados para evitar problemas de overlay
% ═══════════════════════════════════════════════════════════════════════════════
\begin{tikzpicture}[remember picture, overlay]
  % Superior izquierda
  \fill[black] ($(current page.north west)+(8mm,-8mm)$)
    rectangle ++(7mm,-7mm);
  % Superior derecha
  \fill[black] ($(current page.north east)+(-8mm,-8mm)$)
    rectangle ++(-7mm,-7mm);
  % Inferior izquierda
  \fill[black] ($(current page.south west)+(8mm,8mm)$)
    rectangle ++(7mm,7mm);
  % Inferior derecha
  \fill[black] ($(current page.south east)+(-8mm,8mm)$)
    rectangle ++(-7mm,7mm);
\end{tikzpicture}

% ═══════════════════════════════════════════════════════════════════════════════
%  ENCABEZADO DE LA HOJA
% ═══════════════════════════════════════════════════════════════════════════════
\vspace*{8mm}   % espacio para el marcador superior

\noindent
\begin{minipage}[t]{0.57\linewidth}
  {\large\textbf{HOJA DE RESPUESTAS}}\\[3pt]
  {\footnotesize\textbf{Instrucción:} Rellene \textbf{completamente} la burbuja
  con lápiz. Borre completamente si se equivoca.}\\[7pt]
  \textbf{Nombre:}\hrulefill\\[5pt]
  \textbf{Curso:}\underline{\hspace{2.6cm}}\quad
  \textbf{Fecha:}\underline{\hspace{2.6cm}}
\end{minipage}%
\hfill
\begin{minipage}[t]{0.40\linewidth}
  \raggedleft
  {\footnotesize\textbf{N\'{u}mero de identificaci\'{o}n (10 d\'{i}gitos)}}\\[3pt]
  {\tiny 0\,1\,2\,3\,4\,5\,6\,7\,8\,9\quad\textit{(marque un d\'{i}gito por columna)}}\\[2pt]
  \DIG{1}\DIG{2}\DIG{3}\DIG{4}\DIG{5}%
  \DIG{6}\DIG{7}\DIG{8}\DIG{9}\DIG{10}%
\end{minipage}

\vspace{4pt}
\hrule height 0.9pt
\vspace{2pt}

% ─── Leyenda ──────────────────────────────────────────────────────────────────
\noindent
{\footnotesize
\textbf{Correcto:\,}%
\tikz[baseline=-2pt]{\filldraw[fill=black!82,draw=black](0,0)circle(4pt);}%
\qquad
\textbf{Incorrecto:\,}%
\tikz[baseline=-2pt]{%
  \filldraw[fill=black!10,draw=black!50,line width=0.4pt](0,0)circle(4pt);
  \draw[line width=0.8pt,black!60](-2.8pt,-2.8pt)--(2.8pt,2.8pt);}%
\hspace{4pt}%
\tikz[baseline=-2pt]{%
  \filldraw[fill=black!10,draw=black!50,line width=0.4pt](0,0)circle(4pt);
  \draw[line width=0.8pt,black!60](-2.8pt,0)--(2.8pt,0);}%
\hspace{4pt}%
\tikz[baseline=-2pt]{%
  \filldraw[fill=black!30,draw=black!50,line width=0.4pt](0,0)circle(4pt);}%
\,{\tiny marca incompleta}%
\qquad
{\tiny\tikz[baseline=-2pt]{\fill[black](0,-3pt)rectangle(5pt,3pt);} = marca de fila (no escriba sobre ella)}%
}

\vspace{2pt}
\hrule height 0.5pt
\vspace{5pt}

% ═══════════════════════════════════════════════════════════════════════════════
%  BLOQUE DE RESPUESTAS
%  5 columnas de 25 preguntas = 125 preguntas totales
%
%  Cada fila tiene:
%    [TM] [nro].  ○  ○  ○  ○
%    ^                   ^
%    Timing mark         4 burbujas (A B C D)
%
%  El algoritmo OMR:
%   1) Detecta los 4 marcadores fiduciales → corrige perspectiva
%   2) Detecta la columna de timing marks [TM] → calibra Y de cada fila
%   3) Lee el relleno de cada burbuja en posición X fija por columna
%   4) La burbuja con mayor relleno = respuesta seleccionada
% ═══════════════════════════════════════════════════════════════════════════════

\begin{multicols}{5}

\HDR
\Q{1}\Q{2}\Q{3}\Q{4}\Q{5}
\Q{6}\Q{7}\Q{8}\Q{9}\Q{10}
\Q{11}\Q{12}\Q{13}\Q{14}\Q{15}
\Q{16}\Q{17}\Q{18}\Q{19}\Q{20}
\Q{21}\Q{22}\Q{23}\Q{24}\Q{25}

\columnbreak
\HDR
\Q{26}\Q{27}\Q{28}\Q{29}\Q{30}
\Q{31}\Q{32}\Q{33}\Q{34}\Q{35}
\Q{36}\Q{37}\Q{38}\Q{39}\Q{40}
\Q{41}\Q{42}\Q{43}\Q{44}\Q{45}
\Q{46}\Q{47}\Q{48}\Q{49}\Q{50}

\columnbreak
\HDR
\Q{51}\Q{52}\Q{53}\Q{54}\Q{55}
\Q{56}\Q{57}\Q{58}\Q{59}\Q{60}
\Q{61}\Q{62}\Q{63}\Q{64}\Q{65}
\Q{66}\Q{67}\Q{68}\Q{69}\Q{70}
\Q{71}\Q{72}\Q{73}\Q{74}\Q{75}

\columnbreak
\HDR
\Q{76}\Q{77}\Q{78}\Q{79}\Q{80}
\Q{81}\Q{82}\Q{83}\Q{84}\Q{85}
\Q{86}\Q{87}\Q{88}\Q{89}\Q{90}
\Q{91}\Q{92}\Q{93}\Q{94}\Q{95}
\Q{96}\Q{97}\Q{98}\Q{99}\Q{100}

\columnbreak
\HDR
\Q{101}\Q{102}\Q{103}\Q{104}\Q{105}
\Q{106}\Q{107}\Q{108}\Q{109}\Q{110}
\Q{111}\Q{112}\Q{113}\Q{114}\Q{115}
\Q{116}\Q{117}\Q{118}\Q{119}\Q{120}
\Q{121}\Q{122}\Q{123}\Q{124}\Q{125}

\end{multicols}

\end{document}
